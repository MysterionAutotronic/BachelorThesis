% TODO: disable draft mode
\documentclass[11pt, a4paper, oneside, draft]{scrartcl}
\usepackage[T1]{fontenc}
\usepackage[utf8]{inputenc}
\usepackage[ngerman, english]{babel}
\usepackage[onehalfspacing]{setspace}
\usepackage{helvet}
\usepackage[a4paper, left=2.5cm, right=2.5cm]{geometry}
\usepackage[
    backend=biber,
    style=authoryear,
    citestyle=authoryear,
    sorting=nyt,
    sortcites=true,
    dashed=false,
    giveninits=true,
    maxbibnames=999
]{biblatex}
\usepackage{csquotes}
\usepackage{hyperref}
\usepackage{graphicx}
\usepackage{float}
\usepackage{listings}
\usepackage{xcolor}
\usepackage{courier}
\usepackage{needspace}
\usepackage{trace}
\usepackage{longtable}
\usepackage{tabularx}
\usepackage[automark]{scrlayer-scrpage}
\usepackage{microtype}
\usepackage{svg}

% show frames in draft
%\usepackage{showframe}
% Temporary disable citations for better readability while writing
%\renewcommand{\parencite}[2][]{}
%\renewcommand{\textcite}[2][]{}

% set language to english (german styling)
\selectlanguage{english}

% set font to sans-serif
\renewcommand{\familydefault}{\sfdefault}

\addbibresource{jabref-library.bib}

% custom vars
\newcommand{\thesistitle}{Conception, Implementation, and Evaluation of a Highly Scalable and Highly Available Kubernetes-Based SaaS Platform on Kubernetes Control Plane (KCP)}

% metadata
\title{\thesistitle}
\author{David Linhardt}
\date{\today}

\pagestyle{scrheadings}
\cfoot{\thepage} % Page number in center of footer

\begin{document}

    \newgeometry{top=1cm, bottom=1cm, left=2.5cm, right=2.5cm}

    \begin{titlepage}
        \thispagestyle{empty}

        \hspace*{-1.5cm}
        \noindent
        \hfill
        % THI Logo
        \begin{minipage}{0.3\textwidth}
            \raggedleft\
            \hspace*{1cm}
            \includegraphics[width=1\textwidth]{images/thi_logo.pdf}
        \end{minipage}

        \vspace{1cm}

        \hrulefill

        \vspace{4cm}

        % header box
        \noindent
        \makebox[\textwidth][c]{
            \parbox{10cm}{
                \LARGE\textbf{Technische Hochschule Ingolstadt}\\
                \\
                \Large\textbf{Specialist area Computer Science}\\
                \Large\textbf{Bachelor's course Computer Science}\\
            }
        }

        % bachelors thesis
        \begin{center}
            \LARGE\textbf{Bachelor's thesis}
        \end{center}

        \vspace{1cm}

        % tabularx
        \begin{tabularx}{\textwidth}{@{}lX@{}}
            \textbf{Subject:} & \thesistitle \\[2cm]
            \textbf{Name and Surname:} & David Linhardt \\[0.5cm]
            \textbf{Matriculation number:} & 00122706\\[2cm]     
            \textbf{Issued on:} & TODO: Insert Issue Date \\[0.5cm]           
            \textbf{Submitted on:} & TODO: Insert Submit Date \\[2cm]           
            \textbf{First examiner:} & Prof.\ Dr.\ Bernd Hafenrichter \\[0.5cm]      
            \textbf{Second examiner:} & Prof.\ Dr.\ Ludwig Lausser \\
        \end{tabularx}

    \end{titlepage}

    \restoregeometry

    \pagenumbering{roman}

    \section*{Abstract}

    % TOC
    \begingroup
        \microtypesetup{protrusion=false}
        \tableofcontents
    \endgroup

    \section*{Glossary}

    \pagenumbering{arabic}

    \section{Introduction}

        \subsection{Problem Statement and Motivation}



        \subsection{Objectives and Scope}

        \subsection{Structure of the Thesis}

    \section{Fundamentals}

        \subsection{Kubernetes and Multi-Tenancy}

            \paragraph{Kubernetes as the Foundation for Cloud-Native Applications}
                As the de facto standard for deploying and managing 
                \textit{cloud-native applications}, Kubernetes plays a pivotal role in modern cloud
                architecture \parencite[p.~7--8]{poulton2021}.
                Kubernetes works as an application orchestrator for \textit{containerized,
                cloud-native microservice} apps, meaning it can deploy applications and dynamically
                respond to changes \parencite[p.~3]{poulton2021}.
                It offers a platform for declarative configuration and automation for containerized
                workloads, enabling organizations to run distributed applications and services at
                scale \parencite{kubernetesOverview,redhatWhatIsKubernetes}.

            \paragraph{The Importance of Multi-Tenancy in Modern SaaS Platforms}
                Multi-tenancy plays a fundamental role in modern cloud computing. 
                By allowing multiple tenants to share the same infrastructure through
                virtualization, it significantly increases resource utilization, reduces operational
                costs, and enables essential features such as VM mobility and dynamic resource
                allocation \parencite[pp.~345--346]{IEEEMultiTenancySecurityConcerns}. 
                These benefits are critical for cloud providers, as they make the cloud business
                model economically viable and scalable. 
                In the context of modern SaaS platforms, multi-tenancy goes even further by enabling
                unified management, frictionless onboarding, and simplified operational processes
                that allow providers to add new tenants without introducing incremental complexity
                or cost \parencite[pp.~9--11]{awsSaaSArchitectureFundamentals}.
                \\
                However, while multi-tenancy is indispensable for achieving efficiency, scalability,
                and cost-effectiveness, it simultaneously introduces complex security challenges,
                especially in shared environments where resource isolation is limited. 
                In particular, the potential for cross-tenant access and side-channel attacks makes
                security in multi-tenant environments a primary concern
                \parencite[pp.~345--346]{IEEEMultiTenancySecurityConcerns}. 
                As such, understanding and addressing multi-tenancy from both operational and
                security perspectives is essential when designing and securing modern cloud-native
                platforms
                \parencites[pp.~9--11]{awsSaaSArchitectureFundamentals}[p.~4]{isoConcepts}.

            \paragraph{The Challenges of Multi-Tenancy and the Need for Solutions}

            \paragraph{Kubernetes Control Plane (KCP) as a Promising Approach}

            \paragraph{Background: The Evolution of Kubernetes}
                Kubernetes was originally developed at Google and released as open source in 2014
                \parencite{googlecloudWhatIsKubernetes}.

            \paragraph{Background: Containerization as an Enabler of Kubernetes}
                \textit{Containerization} is a way to bundle an application's code with all its
                dependencies to run on any infrastructure thus enhancing portability
                \parencite{awsWhatIsContainerization,dockerWhatContainer}.
                The lightweight nature and isolation can be leveraged by cloud-native software by
                enabling vertical and horizontal autoscaling facilitated by quick container boot
                times, along with self-healing mechanisms and support for distributed, resilient
                infrastructures \parencites{kubernetesAutoscalingWorkloads}{kubernetesSelfHealing}
                    {awsWhatIsContainerization}[pp.~58--59]{davis2019}
                Furthermore it complements the microservice architectural pattern by enabling
                isolated, low overhead deployments, ensuring consistent environments
                \parencite[p.~209]{balalaie2016}.

            \paragraph{Background: The Role of Microservices in Cloud-Native Architectures}
                \textit{Microservices} play a pivotal role in cloud-native architectures by
                promoting agility, scalability, and maintainability of applications.
                By decomposing applications into independent, granular services, microservices
                facilitate development, testing, and deployment using diverse technology stacks,
                enhancing interoperability across platforms
                \parencites[p.~1]{waseem2020}[p.~1]{larrucea2018} and help prevent failures
                in one component from propagating across the system, by isolating functionality into
                distinct, self-contained services \parencite[p.~62]{davis2019}.
                This architectural style aligns well with cloud environments, as it allows services
                to evolve independently, effectively addressing challenges associated with scaling
                and maintenance without being tied to a singular technological framework
                \parencite[pp.~202--203]{balalaie2016}.
                Furthermore, the integration of microservices with platforms like Kubernetes
                enhances deployment automation and orchestration, thus providing substantial
                elasticity to accommodate fluctuating workloads \parencite[p.~170]{haugeland2021}.
                Additionally, migrating legacy applications to microservices can foster
                modernization and efficiency, thus positioning organizations favorably in
                competitive landscapes \parencite[p.~214]{balalaie2016}.
                Overall, the synergy between microservices and cloud-native architectures stems from
                their inherent capability to optimize resource utilization and streamline continuous
                integration and deployment processes.

            \paragraph{Background: Kubernetes Resource Isolation Mechanisms}
                Kubernetes employs several resource isolation mechanisms, primarily through the use
                of \textit{cgroups} (control groups) and \textit{namespaces} to limit resource
                allocation for containers.
                Cgroups are a Linux kernel feature that organizes processes into hierarchical
                groups for fine-grained resource limitation and monitoring via a pseudo-filesystem
                called \textit{cgroupfs} \parencites{kubernetesCgroupsV2}{cgroups7}.
                \textit{Namespaces} are a mechanism for isolating groups of resources withing a
                single cluster and scoping resource names to prevent naming conflicts across
                different teams or projects \parencite{kubernetesNamespaces}.
                However, these mechanisms may not always provide sufficient isolation necessary for
                multi-tenant architectures, because the logical segregation offered by namespaces
                does not address the fundamental security concerns associated with multi-tenancy 
                \parencite[p.~651]{nguyen2022} and research indicates that the native isolation
                strategies can lead to performance interference, where containers that share nodes
                can experience significant degradation in performance due to CPU contention
                \parencite[p.~158]{kim2021}.
                Specifically, critical services may be adversely affected when non-critical services
                monopolize available resources, which undermines the quality of service in
                multi-tenant environments \parencite[p.~30410]{li2019}.
                \\
                Moreover, while Kubernetes allows for container orchestration and resource
                scheduling, it can lead to resource fragmentation, further exacerbating the issue
                of performance isolation \parencite[p.~1]{jian2023}.
                A common approach in multi-tenant scenarios is to deploy separate clusters for each
                tenant, which incurs substantial overhead—particularly in environments utilizing
                virtual machines for isolation \parencite[pp.~144574--144575]{senel2023}.
                In summary, although Kubernetes offers essential isolation mechanisms, the
                complexities of resource sharing and performance consistency in multi-tenant
                applications highlight the need for enhanced strategies to ensure robust resource
                management and performance isolation
                \parencites[p.~651]{nguyen2022}[p.~2]{jian2023}[p.~158]{kim2021}

            \paragraph{Relevance to SaaS and this Thesis}

        \subsection{Kubernetes Control Plane (KCP)}

        \subsection{SaaS Architecture and Automation}

    \section{State of the Art and Related Work}

        \subsection{Zero-Downtime Deployment Strategies}

        \subsection{Kubernetes Scaling Methods}

        \subsection{Multi-Tenancy Concepts in the Cloud}

    \section{Conceptual Design}

        \subsection{System Requirements}

        \subsection{Architecture Design with KCP for SaaS}

        \subsection{Automated Deployment Strategies}

    \section{Prototypical Implementation}

        \subsection{Infrastructure with KCP}

        \subsection[Tenant Provisioning]{Tenant Provisioning (Automation, Multi-Tenancy)}

        \subsection[Scaling Mechanisms]{Scaling Mechanisms (Horizontal Pod Autoscaler)}

        \subsection[Monitoring and Logging]{Monitoring and Logging (Prometheus, Grafana)}

    \section{Evaluation}

        \subsection[Performance Measurements]{Performance Measurements (Downtime, Latency, Scaling)}

        \subsection{Scaling Scenarios \& Optimizations}

        \subsection{Discussion of Results}

        \subsection{Related Work}

    \section{Conclusion and Outlook}

        \subsection{Summary}

        \subsection{Personal Conclusion}

        \subsection{Future Outlook}

    \section*{References}
        \addcontentsline{toc}{section}{References} % manually add to TOC
        \printbibliography[heading=none]

    \section*{List of Figures}
        \addcontentsline{toc}{section}{List of Figures} % manually add to TOC
        \begingroup
            \renewcommand{\listfigurename}{} % remove heading
            \listoffigures
        \endgroup

    \section*{Appendix}

\end{document}
