\documentclass[11pt, a4paper, oneside]{scrartcl}
\usepackage[T1]{fontenc}
\usepackage[utf8]{inputenc}
\usepackage[english]{babel}
\usepackage[onehalfspacing]{setspace}
\usepackage{helvet}
\usepackage[a4paper, left=2.5cm, right=2.5cm]{geometry}
\usepackage[backend=biber, style=authortitle]{biblatex}
\usepackage{csquotes}
\usepackage{hyperref}
\usepackage{graphicx}
\usepackage{float}
\usepackage{listings}
\usepackage{xcolor}
\usepackage{courier}
\usepackage{needspace}
\usepackage{trace}
\usepackage{longtable}
% show page layout
%\usepackage{showframe}

% set font to sans-serif
\renewcommand{\familydefault}{\sfdefault}

\addbibresource{literatur.bib}
\DeclareNameAlias{author}{given-family}
\DeclareNameAlias{labelname}{given-family}

\title{Bachelor Thesis}
\author{David Linhardt}
\date{\today}

% disable linter options
% disable wrong length of dash
% chktex-file 8
% disable interword spacing suggestion
% chktex-file 12
% disable intersentence spacing suggestion
% chktex-file 13
% disable user regex (triggers on \hline)
% chktex-file 44

% listings settings
\lstdefinelanguage{HTML}{
    language=XML,
    sensitive=true,
    alsoletter={<>=-},
    morekeywords={
        div, p, span, h1, h2, h3, body, html, head, title, script, link, meta, style, img, a
    },
    keywordstyle=\color{blue}, % tags
    commentstyle=\color{gray}\itshape,
    stringstyle=\color{red}, % attribs / strings
    morecomment=[s]{<!--}{-->}, % comments
}
\lstdefinelanguage{TS}{
    language=Java,
    sensitive=true,
    alsoletter={.:<>=-},
    morekeywords={
        abstract, as, any, async, await, boolean, break, case, catch, class, const, constructor,
        continue, debugger, declare, default, delete, do, else, enum, export, extends, false,
        finally, for, from, function, get, if, implements, import, in, infer, instanceof, interface,
        is, keyof, let, module, namespace, never, new, null, of, package, private, protected,
        public, readonly, require, return, set, static, string, super, switch, symbol, this, throw,
        true, try, type, typeof, undefined, unique, var, void, while, with, yield
    },
    keywordstyle=\color{blue}, % keywords
    morestring=[b]", % strings % chktex 18
    morestring=[b]', % strings
    stringstyle=\color{red}, % string style
    identifierstyle=\color{black}, % variable style
    comment=[l]//, % single-line comments
    morecomment=[s]{/*}{*/}, % multi-line comments
    commentstyle=\color{gray}\itshape, % comment style
    alsodigit={.}, % includes . as part of numbers
    literate=
        *{=>}{{$\Rightarrow$}}{1}
        {<=}{{$\Leftarrow$}}{1}
        {!=}{{$\neq$}}{1}
        {===}{{$\equiv$}}{1}
        {!==}{{$\not\equiv$}}{1}
        {=>}{{$\Rightarrow$}}{1},
}

\lstset{
    basicstyle=\ttfamily\bfseries\small,
}

% hyphenations
\hyphenation{
    SQL-Da-ten-bank
    NoSQL-Da-ten-bank
    Ja-va-Script
    Type-Script
    Ma-ven
    Log-view-er
    Kamäleon
    Zei-len-um-brü-che
}

\begin{document}

    \sloppy
    \hfuzz=100cm
    \vfuzz=30pt

    \newgeometry{top=1cm, bottom=1cm, left=2.5cm, right=2.5cm}

    \begin{titlepage}
        % BMW Logo
        \hspace*{-1.5cm}
        \noindent
        \begin{minipage}{0.65\textwidth}
            \includegraphics[width=0.2\textwidth]{images/bmw_group_logo.png}
            \includegraphics[width=0.2\textwidth]{images/bmw_logo.png}
            \includegraphics[width=0.2\textwidth]{images/mini_logo.png}
            \includegraphics[width=0.2\textwidth]{images/rolls_royce_logo.png}
        \end{minipage}
        \hfill
        % THI Logo
        \begin{minipage}{0.3\textwidth}
            \raggedleft\
            \hspace*{1cm}
            \includegraphics[width=1\textwidth]{images/thi_logo.jpg}
        \end{minipage}

        \vspace{1.5cm}

        % Titel
        {\Huge\textbf{Praktikantenbericht}}\par
        \vspace{0.5cm}
        {\Large über das praktische Studiensemester}\par
        {\Large im Wintersemester 2023/2024}\par
        \vspace{1cm}

        \hrule

        % Persönliche Daten
        \vspace{0.5cm}
        \begin{tabular}{l@{\hspace{1cm}}l}
            Name, Vorname: & David Linhardt\\
            Studiengang, Semester: & Informatik (Bachelor) 5. Semester\\
            Ausbildungsbetrieb: & BMW Group AG\\
            Straße: & Bremer Straße 6\\
            PLZ, Ort: & 80807 München\\
        \end{tabular}
        \vspace{0.5cm}

        \hrule

        % Dauer
        \vspace{0.5cm}
        \noindent
        \textbf{1 Dauer des Praktikums}\par
        \vspace{0.5em}
        \begin{tabular}{l@{\hspace{0.5cm}}c@{\hspace{0.5cm}}c@{\hspace{0.5cm}}c@{\hspace{0.5cm}}l}
            vom 04.09.2023 bis 05.01.2024 = 18 Wochen \\
        \end{tabular}
        \vspace{1cm}

        % Unterbrechungen
        \noindent
        \textbf{2 Unterbrechungen (Krankheit, Betriebsferien usw.)}\par
        \vspace{0.5em}
        \begin{tabular}{l@{\hspace{0.5cm}}c@{\hspace{0.5cm}}c@{\hspace{0.5cm}}l@{\hspace{0.5cm}}l}
            vom 11.12. bis 15.12. Grund: Krankheit \\
            vom 27.12. bis 29.12. Grund: Betriebsferien* \\
            vom 02.01. bis 05.01. Grund: Betriebsferien* \\
            \multicolumn{5}{l}{
                \textsf{
                    \small{
                        \hspace*{0.1cm}\textit{
                            *Betriebsferien wurden durch Überstunden ausgeglichen
                        }
                    }
                }
            } \\

        \end{tabular}
        \vspace{0.5cm}

        \hrule

        % Unterschriften
        \vspace{0.5cm}
        Die Ausbildungsstelle bestätigt die Angaben des Berichts.\par
        \vspace{1.5cm}
        \noindent
        \begin{tabular}{@{}p{7cm}p{1cm}p{7cm}@{}}
            \dotfill\\
            \multicolumn{1}{l}{(Ort, Datum)}
        \end{tabular}
        \par
        \vspace{2cm}
        \noindent
        \begin{tabular}{@{}p{7cm}p{1cm}p{7cm}@{}}
            \dotfill & & \dotfill \\
            \multicolumn{1}{l}{
                (Unterschrift des Ausbildungsbeauftragten)
            } & & \multicolumn{1}{l}{
                (Firmenstempel)
            }
        \end{tabular}

        % Vertraulichkeit
        \vspace{1.5cm}
        \noindent
        Der Praktikumsbericht wird von der Technischen Hochschule Ingolstadt vertraulich behandelt
        und nur dem /der Beauftragten für das Praktische Studiensemester zur Prüfung und Genehmigung
        weitergeleitet.\par
    \end{titlepage}

    \restoregeometry\

    \section*{Erklärung}
        Ich erkläre hiermit, dass ich den vorliegenden Praxisbericht selbstständig verfasst habe.
        Ich habe keine anderen als die angegebenen Quellen oder Hilfsmittel benutzt.
        Ich habe wörtliche und sinngemäße Zitate als solche gekennzeichnet.\par
        \vspace{1em}
        Ingolstadt, den \today\par
        \vspace{1em}
        \textbf{\underline{Unterschrift des Studenten}}\par
        \vspace*{3cm}
        \begin{tabular}{@{}p{7cm}l@{}}
            \hline
            David Linhardt
        \end{tabular}

        \newpage

    \tableofcontents

        \newpage
    
    \section*{Glossar}
        \addcontentsline{toc}{section}{Glossar} % manually add to TOC
        \setlength{\LTleft}{0pt}  % left align the table
        \setlength{\LTright}{\fill} % ensure the table doesn’t center
        \begin{longtable}{|l|l|} % use longtable to allow page breaks
            % cSpell:disable
            \hline
            \textbf{Abkürzung} & \textbf{Bedeutung}\\
            \hline
            API & \textbf{A}pplication \textbf{P}rogramming \textbf{I}nterface\\
            \hline
            APT & \textbf{A}dvanced \textbf{P}ackage \textbf{T}ool\\
            \hline
            AWS & \textbf{A}mazon \textbf{W}eb \textbf{S}ervices\\
            \hline
            BMW & \textbf{B}ayerische \textbf{M}otoren \textbf{W}erke\\
            \hline
            CD & \textbf{C}ontinuous \textbf{D}eployment\\
            \hline
            CI & \textbf{C}ontinuous \textbf{I}ntegration\\
            \hline
            DB & \textbf{D}aten\textbf{B}ank\\
            \hline
            DEV & \textbf{Dev}elopment\\
            \hline
            DevOps & \textbf{Dev}elopment and \textbf{Op}eration\textbf{s}\\
            \hline
            ECMAScript & \textbf{E}uropean \textbf{C}omputer \textbf{M}anufacturers
                \textbf{A}ssociation \textbf{Script}\\
            \hline
            EKS & \textbf{E}lastic \textbf{K}ubernetes \textbf{S}ervice\\
            \hline
            ELK & \textbf{E}lasticsearch, \textbf{L}ogstash, \textbf{K}ibana\\
            \hline
            ESLint & \textbf{E}CMA\textbf{S}cript \textbf{Lint}er\\
            \hline
            EU & \textbf{E}uropäische \textbf{U}nion\\
            \hline
            FOSS & \textbf{F}ree \textbf{O}pen \textbf{S}ource \textbf{S}oftware\\
            \hline
            GUI & \textbf{G}raphical \textbf{U}ser \textbf{I}nterface\\
            \hline
            INT & \textbf{Int}egration\\
            \hline
            JDBC & \textbf{J}ava \textbf{D}ata\textbf{b}ase \textbf{C}onnectivity\\
            \hline
            Jira & \textbf{J}ira \textbf{I}ssue \textbf{R}eporting \textbf{A}nd \textbf{T}racking
                \textbf{S}ystem\\
            \hline
            K8s & \textbf{K}\textit{ubernete}\textbf{s}\\
            \hline
            Kamäleon & \textbf{Ka}pazitäts\textbf{ma}nag\textbf{e}ment mit
                \textbf{L}ief\textbf{e}ranten \textbf{on}line\\
            \hline
            NPM & \textbf{N}ode \textbf{P}ackage \textbf{M}anager\\
            \hline
            NVM & \textbf{N}ode \textbf{V}ersion \textbf{M}anager\\
            \hline
            NoSQL & \textbf{N}ot \textbf{o}nly \textbf{SQL}\\
            \hline
            OPS & \textbf{Op}eration\textbf{s}\\
            \hline
            PDO & \textbf{P}roduct \textbf{D}evelopment and \textbf{O}perations\\
            \hline
            PR & \textbf{P}ull \textbf{R}equest\\
            \hline
            PROD & \textbf{Prod}uction\\
            \hline
            QA & \textbf{Q}uality \textbf{A}ssurance\\
            \hline
            QMT & \textbf{Q}ualitäts\textbf{m}anagement \textbf{T}eile\\
            \hline
            REST & \textbf{Re}presentational \textbf{S}tate \textbf{T}ransfer\\
            \hline
            SQL & \textbf{S}tructured \textbf{Q}uery \textbf{L}anguage\\
            \hline
            VSCode & \textbf{V}isual \textbf{S}tudio \textbf{Code}\\
            \hline
            WAR & \textbf{W}eb \textbf{A}pplication \textbf{A}rchive\\
            \hline
            WSL2 & \textbf{W}indows \textbf{S}ubsystem for \textbf{L}inux \textbf{2}\\
            \hline
            % cSpell:enable
        \end{longtable}

        \newpage

    \section{Einleitung}
    Im Rahmen meines fünften Studiensemesters habe ich mich dazu entschieden, mein Pflichtpraktikum
    bei der BMW AG abzuleisten.
    Das hatte einige gute Gründe.
    Erst einmal bot das Unternehmen eine offene Praktikantenstelle in der TypeScript
    Softwareentwicklung an, in der ich bis heute auch selbstständig tätig bin und die mir somit eine
    Möglichkeit gab, mich in Sachen Webentwicklung,
    auf die ich mich ohnehin langfristig spezialisieren möchte, weiterzubilden.
    Des Weiteren war die BMW AG für mich als Arbeitgeber wegen der durchaus konkurrenzfähigen
    Vergütung sowie meiner persönlichen Bindung zu den Produkten der Marke durchaus attraktiv,
    da ich selbst mit großem Enthusiasmus BMW fahre.
    Auch der Einsatz in einem großen Projekt mit vielen Stakeholdern und vielen beteiligten
    Entwicklern samt aller einhergehenden Schwierigkeiten war etwas, was ich so aus meinem
    universitären und selbstständig berufstätigen Alltag noch nicht kannte. Kombiniert mit einer 
    35-Stunden Woche und hybridem Arbeiten war auch die Entfernung des Praktikumsbetriebes von
    meinem Wohnort Ingolstadt kein Hindernis mehr.

    \section{Vorstellung des Praktikumsbetriebes}

        \subsection{BMW AG}
        Die Bayerische Motoren Werke AG (BMW AG) ist ein Automobil- und Motorradhersteller mit
        Hauptsitz in München.
        Sie setzt sich aus den Marken \textit{BMW}, \textit{Mini}, \textit{Rolls Royce},
        \textit{BMW M} und \textit{BMW i} zusammen.
        Sie ist mit einem Jahresumsatz von 142,6 Milliarden Euro,
        149.475 Mitarbeitern und fast 2,4 Millionen jährlich abgesetzten Automobilen
        (Stand 2022) eines der größten deutschen Unternehmen und in über 140 Ländern vertreten.
        \textcite{bmw:bericht,bmw:standorte}

        \subsection{Abteilung FG-521}
            \begin{minipage}{\textwidth}
                Eingesetzt wurde ich in der Abteilung \textbf{FG-521}.\par
                Das Abteilungskürzel \textbf{FG-521} steht für:\par
                \textbf{F} -- \textit{Finanzen}\par
                \textbf{FG} -- \textit{Group IT}\par
                \textbf{FG-5} -- \textit{Enabling Prozesse}\par
                \textbf{FG-52} -- \textit{Einkauf, Lieferantennetzwerk}\par
                \textbf{FG-521} -- \textit{IT PDO Volume \& Capacity Management and QMT Frontdesk}
                    \par
            \end{minipage}

        \subsection{Das Projekt Kamäleon}
        % Kamäleon Logo
        \begin{figure}[!htb] % try to place the figure here
            \centering
            \includegraphics[width=0.2\textwidth]{images/kamaeleon_logo.png}
            \caption{Kamäleon Logo}
        \end{figure}
        Konkret war ich im IT-PDO-Capacity-Management-Produkt \textit{Kamäleon} tätig.
        Kamäleon ist ein seit 2017 intern bei BMW entwickeltes Tool zur besseren Nachverfolgung der
        Lieferkette, das Engpässe frühzeitig erkennt, um proaktiv handeln zu können.
        Es soll eine höhere Transparenz schaffen und hat sich gerade während der Zeiten
        unzuverlässiger Halbleiter-Lieferketten in den letzten Jahren bewährt.
        \textcite{cio:kamaeleon}
        Mittlerweile weist das Tool um die 8000 aktive Nutzer sowohl bei BMW als auch bei
        Zulieferern auf.
        Außerdem wird aktuell daran gearbeitet, Kamäleons grundliegende Funktionalität vollständig
        in \textit{Catena-X}, ein von der EU gefördertes offenes Datenökosystem für die
        Wertschöpfungsketten der Automobilindustrie, zu integrieren.
        Catena-X ist Teil des durch die EU geförderten Datenökosystems \textit{Gaia-X}.
        \textcite{catenaX:about,catenaX:compliance}
        Im Speziellen stellte meine Abteilung im Kamäleon-Projekt die Schnittstelle zwischen dem die
        Anforderungen stellenden Fachbereich und den internen sowie externen Entwicklern in
        Südafrika, Bulgarien und Deutschland dar.\par
        Die Grundidee hinter Kamäleon ist, dass dort jeder Zulieferer seine Produktionskapazitäten
        regelmäßig einpflegt und gleichzeitig BMW-Mitarbeiter regelmäßig die Teilebedarfe
        einpflegen, beziehungsweise dass diese regelmäßig bei einer entsprechenden Schnittstelle
        geholt werden.
        Auch können dort sogenannt Kapazitätsfamilien angelegt werden, die aus mehreren Teilen
        bestehen, die  beispielsweise einen Produktionsschritt an einer Maschine gemeinsam haben.
        Falls besagte Maschine also ausfällt, kann schnell festgestellt werden, welche produzierten
        Teile betroffen sind und wie die Versorgung wieder hergestellt werden kann.
        Kamäleon setzt dabei nicht nur beim First-Tier-Supplier (also dem Zulieferer, der das
        fertige Bauteil liefert) an, sondern bildet die gesamte Lieferkette ab.
        Das bedeutet, jeder vorangehende Zulieferer trägt bereits seine eigenen Kapazitäten ein,
        sodass es beim nachfolgenden Glied in der Lieferkette nicht zu einem Versorgungsfall kommt.
        \par
        % Kamäleon Screenshot
        \begin{figure}[H] % fix position
            \centering
            \includegraphics[width=1\textwidth]{images/kamaeleon_screenshot.png}
            \caption{Screenshot von Kamäleon}
        \end{figure}

    \section[Hauptteil]{Hauptteil: Beschreibung der einzelnen Projekte, Tätigkeiten und Aufgaben}

        \subsection{Grundliegende Architektur}
            % Architekturdiagramm
            \begin{figure}[!htb] % try to place the figure here
                \centering
                \includegraphics[width=0.8\textwidth]{images/architekturdiagramm.png}
                \caption{Architekturdiagramm}
            \end{figure}
            Kamäleon besteht im Prinzip aus einem Frontend, das in Angular geschrieben ist, und
            einem Backend, das in Java geschrieben ist.
            Das Frontend kommuniziert über REST-API-Calls mit dem Backend, das wiederum über JDBC
            auf die Datenbanken zugreift.
            Die Datenbanken sind in Docker-Containern angesiedelt und laufen auf einem
            Kubernetes-Cluster in der AWS-Cloud.
            Interessant ist, dass es zwei Datenbanken gibt, eine SQL-Datenbank und eine
            NoSQL-Datenbank.
            Die SQL-Datenbank wird für die Stammdaten verwendet, die NoSQL-Datenbank für die
            Transaktionsdaten.
            Es handelt sich hier also um eine hybride Datenbankarchitektur.
            Während die SQL-Datenbank auf PostgreSQL basiert, basiert die NoSQL-Datenbank auf dem
            ELK-Stack.\par
            Interessant ist auch die Branching-Strategie, die in Kamäleon verwendet wird.
            Diese zeichnet sich durch eine klare Struktur und eine hohe Flexibilität aus,
            um den Entwicklungsprozess mit verschiedenen internationalen Teams zu koordinieren.
            Besonders hervorzuheben ist hierbei die Verwendung von zwei DEV-Branches,
            die parallel laufen und von zwei verschiedenen Teams betreut werden.
            Jedes Team hat also seinen eigenen DEV-Branch, in dem es arbeitet.
            Dies ist notwendig, da in dem Projekt ein DevOps-Provider in Südafrika und ein
            Dev-Provider in Bulgarien beteiligt sind, die sich sonst durch Downtimes der
            DEV-Umgebung gegenseitig ausbremsen würden.

            % Git Strategy
            \begin{figure}[H] % fix position
                \centering
                \includegraphics[width=1\textwidth]{images/gitgraphCropped.png}
                \caption{Git-Strategy}
            \end{figure}

            Jeder Entwickler arbeitet an einem Ticket aus dem Jira-Board, das ihm zugewiesen wurde.
            Dazu erstellt er einen neuen feature-Branch auf Basis des PROD-Branches, den er nach der
            Ticketnummer benennt,
            arbeitet an dem Ticket und testet seine Änderungen in der lokalen Entwicklungsumgebung.
            Wenn die Anforderungen des Tickets erfüllt sind, erstellt er einen PR, der von einem
            Senior-Entwickler reviewed wird.
            Wenn der PR genehmigt wird, wird er in den DEV-Branch gemerged.
            Dort kann der Entwickler dann seine Änderungen noch einmal auf einer Live-Umgebung
            testen.
            Wenn das Feature dort ebenfalls funktioniert, wird es in den INT-Branch gemerged, wo es
            von den QA-Testern getestet wird.
            Auf den INT-Branch dürfen drei Tage vor dem Release keine Änderungen mehr gemacht
            werden.
            Wenn die QA-Tester den aktuellen Stand des INT-Branches absegnen,
            wird der INT-Branch in den PROD-Branch gemerged und das Feature ist live.
            Anschließend werden die beiden DEV-Branches ebenfalls auf den Stand des INT-Branches
            gebracht.\par
            Hotfixes hingegen werden nur auf den DEV-Branches getestet und dann direkt per
            Cherry-Pick auf den PROD-Branch gebracht.
            Das ermöglicht es, Hotfixes schnell in die Produktivumgebung zu bringen.

        \subsection{Einarbeitung}
            \subsubsection{Onboarding}
            Wie in jeder neuen Abteilung musste ich mich auch in meiner erst einmal einarbeiten.
            Dazu mussten natürlich erst einmal Laptop und Mobiltelefon eingerichtet werden,
            Zugänge für diverse Unternehmensressourcen angefordert und autorisiert werden,
            einige Schulungen und eLearning-Module zu den Themen Datenschutz, Informationssicherheit
            und Sicherheit am Arbeitsplatz absolviert werden und natürlich die lokale
            Entwicklungsumgebung aufgesetzt werden, die ich in den nächsten Monaten nutzen würde.
            Besonders das Anfordern der Zugänge war ein recht langwieriger Prozess,
            da die Anträge erst von mehreren Stellen genehmigt werden mussten und ich Zugriff auf
            viele verschiedene Systeme und Tools benötigte.
            Außerdem musste ich einige Schulungen zu Kamäleon selbst absolvieren,
            um die Funktionsweise des Tools und die Anforderungen des Fachbereichs zu verstehen.
            Zu guter Letzt war es allerdings auch notwendig, eine Excel-Tabelle aufzusetzen, in die
            ich meine Arbeitszeiten und Tätigkeiten eintragen würde, damit auch die Stunden korrekt
            abgeleistet werden konnten.

            \subsubsection{Aufsetzen der Entwicklungsumgebung}
            Da die lokale Entwicklungsumgebung komplex war, gestaltete sich das Einrichten dieser
            als herausfordernd.
            Nach dem problemlosen Einrichten der Proxy-Konfiguration und dem Setzen der
            Umgebungsvariablen traten währen der Einrichtung einige Bugs auf,
            die durch inkonsistente Konfigurationen und fehlende Abhängigkeiten verursacht wurden.
            Diese Bugs mussten erst einmal behoben werden, bevor ich mit der eigentlichen Arbeit
            beginnen konnte.\par
            Unter Anderem schlug das Deployment der gebuildeten Anwendung als WAR-Datei auf den
            Payara-Server mehrmals, mitunter nach langen Wartezeiten fehl.\par
            Auch die Einrichtung der Docker-Container für die beiden Datenbanken stellte sich als
            schwierig heraus, da hier Versionskonflikte auftraten.
            Nach einigen Fixes in der Dockerfile und dem Anpassen der lokalen SQL-Datenbank im
            Docker-Container mittels Flyway (einem Tool zur Versionskontrolle von Datenbanken)
            konnte ich das Backend erfolgreich starten und mit dem Aufsetzen des Frontends beginnen.
            \par
            Das Einrichten des Frontends gestaltete sich weniger schwierig, allerdings musste ich
            hier auch einige Anpassungen machen.
            Durch den Einsatz eines Authenticated-Proxies musste ich die NPM-Konfiguration (.npmrc)
            anpassen und die NPM-Registry abändern, um die benötigten Pakete herunterladen zu
            können.
            Nach dem NPM-Install kam es allerdings zu einigen weiteren Fehlern, diese konnten
            allerdings durch ein Update der tsconfig-Datei auf einen neueren ECMAScript-Standard von
            ES2020 auf ES2022 behoben werden.\par
            Nachdem ich die Entwicklungsumgebung erfolgreich aufgesetzt hatte, stieß ich auf ein
            weiteres Problem:
            Der Laptop, den ich von der BMW AG gestellt bekommen hatte,
            war nicht leistungsfähig genug,
            um die Entwicklungsumgebung flüssig laufen zu lassen.
            Nach einem Upgrade des Arbeitsspeichers von 16GB auf 32GB konnte ich erst mit der
            eigentlichen Arbeit beginnen.\par
            Während des gesamten Prozesses pflegte ich die gefundenen Lösungen auf Confluence ein,
            damit zukünftige Praktikanten und neue Mitarbeiter im Projekt von meinen Erfahrungen
            profitieren konnten, und ergänzte den mittlerweile veralteten Guide, der die Einrichtung
            der Entwicklungsumgebung Schritt für Schritt erklärte.

        \subsection{Linting und Konfiguration}
            \subsubsection{Editorconfig}
            Um die Codequalität zu erhöhen und die Codebasis konsistent zu halten, durfte ich in
            Kamäleon eine Editorconfig einrichten.
            Diese definiert einheitliche Einstellungen für die Entwicklungsumgebung, wie
            beispielsweise die Einrückungstiefe, die Zeilenlänge und die Zeilenumbrüche.
            Dadurch wird sichergestellt, dass alle Entwickler in der gleichen Umgebung arbeiten und
            der Code konsistent bleibt.
            Die Editorconfig wurde in das Repository eingecheckt und von allen Entwicklern in der
            Entwicklungsumgebung eingebunden.

            \subsection{CI/CD-Pipeline}
            Zu dem Zeitpunkt, als ich dem Kamäleon-Team beitrat, gab es bereits eine CI/CD-Pipeline,
            die in Jenkins eingerichtet war.
            Diese Pipeline war in der Lage, die Anwendung zu bauen und auf den Payara-Server zu
            deployen.
            Durch eine Umstellung weg von der Atlassian-Toolchain hin zu GitHub Enterprise und damit
            auch zu GitHub Actions musste die Pipeline allerdings migriert werden.
            In diesen Prozess eingebunden zu sein war eine spannende Erfahrung,
            da ich so die Möglichkeit hatte, an der Konfiguration der neuen Pipeline mitzuwirken.

            \subsubsection{SonarQube}
            Da BMW bereits eine vorgefertigte SonarQube-Instanz betrieb,
            wurde diese auch in Kamäleon eingesetzt.
            SonarQube ist ein statisches Codeanalyse-Tool, das in der Lage ist,
            Fehler und potenzielle Probleme im Code zu finden.
            Es kann auch Code-Stilregeln überprüfen und sicherstellen,
            dass der Code Best-Practice-Standards entspricht.
            Um den Entwicklern Feedback über die Qualität ihres Codes zu geben,
            wurde SonarQube nicht nur in die CI/CD-Pipeline eingebunden,
            sondern auch in die Entwicklungsumgebung für das Backend, in diesem Fall IntelliJ IDEA.

            \subsubsection{ESLint}
            Um die Codequalität weiter zu erhöhen, durfte ich in Kamäleon außerdem eine
            ESLint-Konfiguration einrichten.
            ESLint ist ein weiteres statisches Codeanalyse-Tool,
            allerdings spezialisiert auf JavaScript und TypeScript.
            Da ich den Vorschlag eingebracht hatte ESLint in Kamäleon zu verwenden,
            durfte ich auch die Konfiguration in Gänze eigenständig übernehmen.
            Wichtig für die Liner Konfiguration in einem bereits bestehenden Projekt war es,
            die Konfiguration so zu wählen,
            dass die bestehende Codebasis nicht zu stark verändert werden musste,
            aber trotzdem die Codequalität erhöht wurde.
            Besonders das direkte Feedback in der Entwicklungsumgebung bietet einen großen Mehrwert,
            da so stilistische Fehler direkt beim Schreiben des Codes erkannt und behoben werden
            können.

            \subsubsection{Script}
            Anschließend an die Einrichtung von ESLint und nach einiger Arbeit am Frontend
            durfte ich, da ein neuer Dev-Provider ins Projekt kommen sollte,
            in Zusammenarbeit mit einem weiteren Praktikanten ein Powershell-Script schreiben,
            das die benötigten Standardtools installiert, die Umgebungsvariablen setzt und die
            Entwicklungsumgebung konfiguriert.
            Um die Installation der Tools zu vereinfachen, verwendeten wir Chocolatey,
            einen Paketmanager für Windows.
            Dieser ermöglicht die einfache Installation von vielen Tools,
            die in der Entwicklungsumgebung benötigt werden, ähnlich zu APT unter Linux.\par
            Die zu installierenden Tools umfassten unter anderem NVM zur Verwaltung der
            Node-Versionen für die verschiedenen Node-Server, Node.js, WSL2, px
            (ein Proxy-Wrapper zur Authentifizierung beim Corporate-Proxy), Git, Sourcetree
            (ein GUI für Git), Docker-Desktop, IntelliJ IDEA, VSCode
            (samt der Extensions für ESLint, Editorconfig und cSpell, ein Spellchecker für VSCode),
            SoapUI (eine FOSS-Alternative zu Postman), SnakeTail (ein Logviewer), Java, Maven und
            einige mehr.
            Das Script sollte also die gesamte Entwicklungsumgebung aufsetzen und konfigurieren,
            sodass die neuen Entwickler direkt mit der Arbeit beginnen konnten.

        \subsection{Documentation}
        Während meiner Zeit im Kamäleon Team pflegte ich außerdem die Dokumentationen, immer wenn
        mir in diesen eine veraltete Passage auffiel oder ich eine neue Lösung für ein Problem
        gefunden hatte.
        Primär war das die Confluence-Seite zum Aufsetzen der Entwicklungsumgebung,
        ferner aber auch eine Liste an hilfreichen Tools sowie ein Guide für Praktikanten,
        die nach mir neu ins Team kamen.
        Vor allem die dann gut gepflegte Confluence-Seite und die Liste nützlicher Tools stellten
        sich später als sehr hilfreich heraus, da sie später als Basis für das Script dienten,
        das die Entwicklungsumgebung nun automatisiert aufsetzen sollte.

        \subsection{Übersetzungen}
        Die meiste Arbeit an Kamäleon bestand allerdings in einem sehr großen Ticket,
        das diverse Fehler, fehlende Übersetzungen und Unstimmigkeiten im Frontend adressierte.
        In diesem Ticket wurden quasi alle Darstellungsfehler, fehlenden Übersetzungen und
        Unstimmigkeiten im Frontend zusammengetragen.
            \subsubsection{ngx-translate}
            Kamäleon nutzt zur Internationalisierung die Bibliothek ngx-translate, die es
            ermöglicht, die Anwendung in mehreren Sprachen darzustellen, in diesem Fall Deutsch und
            Englisch.\par
            Ein großer Teil des Tickets bestand aus diversen Tipp- und Rechtschreibfehlern, die sich
            in den Übersetzungsdateien fanden.
            Um nicht jeden Fehler einzeln suchen zu müssen und möglicherweise einige nicht im
            Ticket enthaltene Fehler zu übersehen, nutzte ich die VSCode-Erweiterung cSpell, die
            einen Spellchecker für VSCode bereitstellt.
            Mit dieser Erweiterung wurden mir nun in der doch recht umfangreichen Übersetzungsdatei
            Hunderte von Fehlern angezeigt.
            Nun musste ich nur noch die Fehler beheben beziehungsweise fälschlicherweise als Fehler
            erkannte Wörter zu einem Wörterbuch im Repository hinzufügen.\par
            Etwas schwieriger verhielt es sich mit den falschen Übersetzungen, die sich in der Datei
            fanden.
            Besonders mit jenen, die sich kontextabhängig unterschieden.
            Hier war es mitunter notwendig, mehrere Variablen für ein einziges englisches Wort zu
            verwenden, um die im jeweiligen Kontext korrekte Übersetzung zu finden.
            Interessant jedoch wurde es häufig bei den fehlenden Übersetzungen.
            Oftmals waren diese nicht einfach vergessen worden, sondern die üblichen Wege der
            Übersetzung, wie beispielsweise die Verwendung von ngx-translate-Pipes, funktionierte an
            diesen Stellen nicht.\par
            \begin{figure}[h!]
                \centering
                \begin{lstlisting}[language=HTML, frame=single, basicstyle=\ttfamily\small]
                    <p>{{ 'NAMESPACE.KEY' | translate }}</p>
                \end{lstlisting}
                \caption{ngx-translate-Pipe}
            \end{figure}
            Häufig war es hier notwendig, einen Helper Service zu nutzen, der als Parameter den Key
            der Übersetzung bekam und die Übersetzung für die gerade eingestellte Sprache zurückgab.
            An einigen Stellen renderte Angular allerdings auch mit dem Service die Übersetzung
            nicht korrekt und es war notwendig, Operationen zu tätigen, die ein Neurendern des
            Templates erzwangen.
            Da es sich um eine frühe Version von Angular handelte, waren auch Signals und die neue
            \textit{@for}-Schleife noch nicht verfügbar, wodurch für Arrays häufig ein Neurendern
            erzwungen werden musste, wenn sich einzelne Elemente änderten.

            \subsubsection{AgGrid}
            Ein großer Teil der fehlenden Übersetzungen befand sich außerdem in den
            AgGrid-Komponenten, die in Kamäleon verwendet werden.
            AgGrid ist eine Bibliothek für Tabellen, die sehr mächtig und umfangreich  ist.
            AgGrid benötigte ein eigenes Übersetzungsfile, das die Übersetzungen für die
            AgGrid-Komponenten enthielt.
            Dieses erstellte ich neu.
            Es handelt sich hierbei um eine einzelne Konstante, die eine Map enthält, die die
            Keys auf die jeweiligen Übersetzungen abbildet.
            Diese Konstante wurde in jede Tabelle als localeText-Property übergeben, wodurch die
            Übersetzungen in den Tabellen angezeigt wurden.\par
            \begin{figure}[h!]
                \centering
                \begin{lstlisting}[language=TS, frame=single, basicstyle=\ttfamily\small]
                    const gridOptions = {
                        localeText: AG_GRID_LOCALE_DE,

                        // other grid options ...
                    }
                \end{lstlisting}
                \caption{AgGrid-localeText-Property}
            \end{figure}
            Zum Wechseln der Sprache war es notwendig, die localeText-Property neu zu setzen und die
            Tabelle neu zu rendern.
            Da es zu dem Zeitpunkt, an dem ich in Kamäleon arbeitete, noch keine vorgefertigte
            locale-Datei in deutscher Sprache gab, musste ich diese selbst erstellen.
            Da die locale-Datei allerdings um die 500 Keys enthielt, dauerte auch das eine Weile.
            Durch diese Arbeit wurde allerdings jedes mögliche AgGrid-Element auch in zukünftigen
            Tabellen übersetzt.\par
            Insgesamt umfasste das Ticket, als ich es fertig gestellt hatte um die 16.000 Zeilen
            Code in etwa 300 Dateien.
            Ich hatte also quasi jede Datei in Frontend einmal durchgesehen und bearbeitet,
            wodurch ich einen sehr guten Überblick über die gesamte Codebasis bekam.
            % Kamäleon Screenshot mit Markierungen
            \begin{figure}[H] % fix position
                \centering
                \includegraphics[width=1\textwidth]{images/kamaeleon_screenshot_edited.png}
                \caption{Screenshot von Kamäleon mit Markierungen an von mir veränderten Stellen} 
            \end{figure}

    \section{Zusammenfassung}
    Bei der Arbeit an Kamäleon konnte ich mein bereits vorhandenes Wissen in der Webentwicklung
    vertiefen und erweitern.
    Auch die Arbeit in einem internationalen Team und die Arbeit an einem großen Projekt mit vielen
    Stakeholdern war eine spannende Erfahrung, die mir gezeigt hat, wie wichtig eine gute
    Kommunikation und eine klare Struktur in einem Projekt sind.
    Besonders die Herausforderungen, die durch eine große Zahl an gleichzeitig arbeitenden
    Entwicklern entstehen, haben mir gezeigt, wie wichtig eine gute Branching-Strategie und eine
    klare Dokumentation sowie einheitliche Konfigurationen und eine CI/CD-Pipeline sind.
    Dies ermöglichte mir, im Projekt im 6. Semester eine klare Struktur zu schaffen.
    Auch mit einigen anderen Themen wie Containerisierung, Internationalisierung und vor allem dem
    Issue-Workflow hatte ich zuvor wenig Anknüpfungspunkte und konnte somit viel aus dieser
    Erfahrung profitieren.
    Auch wenn meine Arbeit an Kamäleon sich zu einem großen Teil auf ein Ticket im Frontend
    beschränkte, konnte ich doch viele verschiedene Aspekte der Softwareentwicklung kennenlernen und
    anwenden und so mein Wissen in der Praxis vertiefen.

    \section{Persönliches Fazit}
    Insgesamt war mein Praktikum bei der BMW AG eine sehr lehrreiche und spannende Erfahrung.
    Ich konnte viele neue Aspekte der Softwareentwicklung kennenlernen und anwenden und so mein
    Wissen in der Praxis vertiefen.
    Das Team war sehr nett und hilfsbereit, hat mich gut in das Projekt eingeführt und mir bei
    Fragen immer zur Seite gestanden.
    Auch die Arbeitsatmosphäre war sehr angenehm, und ich wurde nicht nur im Büro voll in das Team
    integriert, sondern auch außerhalb des Büros bei Teamevents und gemeinsamen Mittagessen.
    Ich bekam sogar eine Führung durch das für die Öffentlichkeit nicht zugängliche Werk 0, in dem
    Prototypen und Vorserienfahrzeuge gefertigt werden.
    Durch einen Besuch einiger Entwickler aus Südafrika konnte ich in persönlichen Gesprächen auch
    die Kultur und Arbeitsweise in anderen Ländern kennenlernen und so meinen Horizont erweitern.
    Ich würde die BMW AG als Arbeitgeber und das Kamäleon-Team als Arbeitsumfeld auf jeden Fall
    weiterempfehlen und bin sehr froh, dass ich die Möglichkeit hatte, mein Praktikum dort
    in einem tollen Team zu absolvieren.
    Auch durch die Möglichkeit, an allen Meetings teilzunehmen, bekam ich eine andere Perspektive
    auf die Softwareentwicklung und die betriebswirtschaftlichen Zusammenhänge.
    Auch hat mir der Austausch mit den Kolleginnen und Kollegen ein besseres Bild über
    Karrieremöglichkeiten und -chancen in der Automobilbranche vermittelt und mir die Möglichkeit
    gegeben, mich besser auf meine berufliche Zukunft vorzubereiten.

    \newpage

    \section*{Abbildungsverzeichnis}
        \addcontentsline{toc}{section}{Abbildungsverzeichnis} % manually add to TOC
        \begingroup
            \renewcommand{\listfigurename}{} % remove heading
            \listoffigures
        \endgroup

    \section*{Literaturverzeichnis}
        \addcontentsline{toc}{section}{Literaturverzeichnis} % manually add to TOC
        \printbibliography[heading=none]

        \LaTeX\

\end{document}
