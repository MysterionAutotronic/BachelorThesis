% TODO: disable draft mode
\documentclass[11pt, a4paper, oneside, draft]{scrartcl}
\usepackage[T1]{fontenc}
\usepackage[utf8]{inputenc}
\usepackage[ngerman, english]{babel}
\usepackage[onehalfspacing]{setspace}
\usepackage{helvet}
\usepackage[a4paper, left=2.5cm, right=2.5cm]{geometry}
\usepackage[backend=biber, style=authortitle]{biblatex}
\usepackage{csquotes}
\usepackage{hyperref}
\usepackage{graphicx}
\usepackage{float}
\usepackage{listings}
\usepackage{xcolor}
\usepackage{courier}
\usepackage{needspace}
\usepackage{trace}
\usepackage{longtable}
\usepackage{tabularx}
\usepackage[automark]{scrlayer-scrpage}
\usepackage{microtype}
\usepackage{svg}
% show frames in draft
%\usepackage{showframe}
% Temporary disable citations for better readability while writing
\renewcommand{\parencite}[2][]{}
\renewcommand{\textcite}[2][]{}

% set language to english (german styling)
\selectlanguage{english}

% set font to sans-serif
\renewcommand{\familydefault}{\sfdefault}

\addbibresource{jabref-library.bib}
\DeclareNameAlias{author}{given-family}
\DeclareNameAlias{labelname}{given-family}

% custom vars
\newcommand{\thesistitle}{Conception, Implementation, and Evaluation of a Highly Scalable and Highly Available Kubernetes-Based SaaS Platform on Kubernetes Control Plane (KCP)}

% metadata
\title{\thesistitle}
\author{David Linhardt}
\date{\today}

\pagestyle{scrheadings}
\cfoot{\thepage} % Page number in center of footer

\begin{document}

    \newgeometry{top=1cm, bottom=1cm, left=2.5cm, right=2.5cm}

    \begin{titlepage}
        \thispagestyle{empty}

        \hspace*{-1.5cm}
        \noindent
        \hfill
        % THI Logo
        \begin{minipage}{0.3\textwidth}
            \raggedleft\
            \hspace*{1cm}
            \includegraphics[width=1\textwidth]{images/thi_logo.pdf}
        \end{minipage}

        \vspace{1cm}

        \hrulefill

        \vspace{4cm}

        % header box
        \noindent
        \makebox[\textwidth][c]{
            \parbox{10cm}{
                \LARGE\textbf{Technische Hochschule Ingolstadt}\\
                \\
                \Large\textbf{Specialist area Computer Science}\\
                \Large\textbf{Bachelor's course Computer Science}\\
            }
        }

        % bachelors thesis
        \begin{center}
            \LARGE\textbf{Bachelor's thesis}
        \end{center}

        \vspace{1cm}

        % tabularx
        \begin{tabularx}{\textwidth}{@{}lX@{}}
            \textbf{Subject:} & \thesistitle \\[2cm]
            \textbf{Name and Surname:} & David Linhardt \\[2cm]     
            \textbf{Issued on:} & TODO: Insert Issue Date \\[0.5cm]           
            \textbf{Submitted on:} & TODO: Insert Submit Date \\[2cm]           
            \textbf{First examiner:} & Prof.\ Dr.\ Bernd Hafenrichter \\[0.5cm]      
            \textbf{Second examiner:} & Prof.\ Dr.\ Ludwig Lausser \\
        \end{tabularx}

    \end{titlepage}

    \restoregeometry

    \pagenumbering{roman}

    \section*{Abstract}

    % TOC
    \begingroup
        \microtypesetup{protrusion=false}
        \tableofcontents
    \endgroup

    \section*{Glossary}

    \pagenumbering{arabic}

    \section{Introduction}

        \subsection{Problem Statement and Motivation}



        \subsection{Objectives and Scope}

        \subsection{Structure of the Thesis}

    \section{Fundamentals}

        \subsection{Kubernetes and Multi-Tenancy}

            \paragraph{Introduction and Motivation}
                % Why Kubernetes matters
                As the de facto standard for deploying and managing 
                \textit{cloud-native applications}, Kubernetes plays a pivotal role in modern cloud
                architecture \parencite[p.~7--8]{poulton2021}.
                Kubernetes works as an application orchestrator for \textit{containerized,
                cloud-native microservice} apps, meaning it can deploy applications and dynamically
                respond to changes \parencite[p.~3]{poulton2021}.
                It offers a platform for declarative configuration and automation for containerized
                workloads, enabling organizations to run distributed applications and services at
                scale \parencite{kubernetesOverview,redhatWhatIsKubernetes}.

                % Why multi-tenancy matters in modern SaaS-platforms
                Multi-tenancy plays a fundamental role in modern cloud computing.
                By allowing multiple tenants to share the same infrastructure through
                virtualization, it significantly increases resource utilization, reduces operational
                costs, and enables essential features such as VM mobility and dynamic resource
                allocation \parencite[pp.~345--346]{IEEEMultiTenancySecurityConcerns}.
                These benefits are critical for cloud providers, as they make the cloud business
                model economically viable and scalable.
                \\
                However, while multi-tenancy is indispensable for efficiency and cost-effectiveness,
                it simultaneously introduces complex security challenges, especially in shared
                environments where resource isolation is limited. As such, understanding and
                addressing multi-tenancy is essential when designing and securing modern
                cloud-native platforms \parencite[pp.~345--346]{IEEEMultiTenancySecurityConcerns}.

            \paragraph{Overview of Kubernetes}
                % History
                Kubernetes was originally developed at Google and released as open source in 2014
                \parencite{googlecloudWhatIsKubernetes}.
                % Containerization
                \textit{Containerization} is a way to bundle an application's code with all its
                dependencies to run on any infrastructure thus enhancing portability
                \parencite{awsWhatIsContainerization,dockerWhatContainer}.
                This comes with additional advantages that can be leveraged by Kubernetes, including
                vertical and horizontal autoscaling facilitated by quick container boot times, along
                with self-healing mechanisms and support for distributed, resilient infrastructures
                \parencite{kubernetesAutoscalingWorkloads,kubernetesSelfHealing,
                    awsWhatIsContainerization}.

            \paragraph{Kubernetes Resource Isolation Mechanisms}

            \paragraph{Multi-Tenancy Challenges in Kubernetes}

            \paragraph{Approaches to Multi-Tenancy in Kubernetes}

            \paragraph{Relevance to SaaS and this Thesis}

        \subsection{Kubernetes Control Plane (KCP)}

        \subsection{SaaS Architecture and Automation}

    \section{State of the Art and Related Work}

        \subsection{Zero-Downtime Deployment Strategies}

        \subsection{Kubernetes Scaling Methods}

        \subsection{Multi-Tenancy Concepts in the Cloud}

    \section{Conceptual Design}

        \subsection{System Requirements}

        \subsection{Architecture Design with KCP for SaaS}

        \subsection{Automated Deployment Strategies}

    \section{Prototypical Implementation}

        \subsection{Infrastructure with KCP}

        \subsection[Tenant Provisioning]{Tenant Provisioning (Automation, Multi-Tenancy)}

        \subsection[Scaling Mechanisms]{Scaling Mechanisms (Horizontal Pod Autoscaler)}

        \subsection[Monitoring and Logging]{Monitoring and Logging (Prometheus, Grafana)}

    \section{Evaluation}

        \subsection[Performance Measurements]{Performance Measurements (Downtime, Latency, Scaling)}

        \subsection{Scaling Scenarios \& Optimizations}

        \subsection{Discussion of Results}

        \subsection{Related Work}

    \section{Conclusion and Outlook}

        \subsection{Summary}

        \subsection{Personal Conclusion}

        \subsection{Future Outlook}

    \section*{References}
        \addcontentsline{toc}{section}{References} % manually add to TOC
        \printbibliography[heading=none]

    \section*{List of Figures}
        \addcontentsline{toc}{section}{List of Figures} % manually add to TOC
        \begingroup
            \renewcommand{\listfigurename}{} % remove heading
            \listoffigures
        \endgroup

    \section*{Appendix}

\end{document}
